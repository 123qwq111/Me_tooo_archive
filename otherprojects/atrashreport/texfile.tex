\documentclass{ctexart}%类型

\usepackage{amsmath,lmodern}%宏包

\begin{document}
\part*{升力的简化解释}
在正式开始讨论升力简化解释的目的、作用之前,我想先介绍两个例子,之后的讨论方向也会围绕这两个例子展开。\par
第一个例子是一个众所周知的数学问题:$1,\dot{0.9}$。按照时间顺序,大概在小学阶段,按照$0<1$,会得到$1>\dot{0.9}$的结果。过了几年,看科普视频,遇到戴德金分割法证明$1=\dot{0.9}$,就向别人炫耀。前一段时间,看到超实数和非标准分析的文章后,认识到在实数范围内$1=\dot{0.9}$,在超实数范围内,$1>\dot{0.9}$。\par
第二个例子是关于化学的。对于物质结构的刻画,有价层电子对互斥理论、杂化理论、分子轨道理论。以氧分子为例,价层电子对互斥理论可以解释氧分子直线型,但对于氧分子的顺磁性,只有分子轨道理论能解释。对于晶体类型,以成分多少来规定物质的晶体类型。\par
从第一个例子,可以看出漂石理论、流管变化理论相当于前两种情况,它们都是更全面的解释的一个方面,只不过学的有限,要学的也没那么深入。毕竟如果在小学一年级讲环量、集合论,那我就很怀疑到底要怎么教、怎么学、怎么考了。因此升力的简化解释可以不全包括所有的要点,甚至可以在一段时间后被新学的推翻,核心在于扫“盲”,不能致“盲”,只不过飞机升力的来源至今实际上远比我们目前学习的知识复杂,科学家至今都没能提出一个完美的理论解释(文章编号:(1008-41342022)12-0016-05)。\par
从第二个例子,可以得出完美的理论并非必要,毕竟拿计算机模拟需要耗时间,典型的吃力不讨好,还晦涩难懂。典型的例子就是流体力学的量子化,分子数会导致完美的计算爆炸,最后还要分区域。简化解释就是不吃力还讨好的方法。除此之外,还能得到的结论是:要尽量找影响程度大的因素。\par
不过对于等时理论,作为很多人在义务教育阶段会学到的,可能之后再也不会遇到这类问题,显然应该被抛弃。类似的有:高铁的黄线。如果模拟的话,得出的结果是:列车朝人驶来时会产生非常大的力,方向为指向车头,但列车经过时,力明显减小,越到后面越大,方向与行驶方向相同。还有,吹纸带的问题。水平吹时,由伯努利方程计算出的力远小于纸的重力。往好里想,反差说不定更印象深刻呢。\par
对于简化解释,我的观点是在不违背实验的前提下,可以是影响程度小的方面,当然影响程度大更好。
\end{document}
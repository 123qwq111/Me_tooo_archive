%--------------------------------------------------
	%	第一章
	\chapter{网络舆情及其部分特点}
	
	
	%-----------------------------------------------
	\section{网络舆情的定义}
	网络舆情是由于各种事件的刺激而产生的、通过互联网传播的、人们对于该事件的所有认知、态度、情感和行为倾向的集合。\upcite{TSQB200918023}舆论简单的定义是:是社会中相当数量的人对于一个特定话题所表达的个人观点、态度和信念的集合体。网络舆情可以发展为舆论,因此研究网络舆情也要研究相应的网络舆论。\par
	\section{网络舆情的特点}
	片面性、滞后性是网络舆情的一个特点。部分片面、滞后是由于某些信息没有在互联网上公开,因此人们的行为、情感倾向可能与事实不符甚至相反,比如天津中德贫困生助学金事件。事发后第一时间许多人为当事人打抱不平,但之后事情却发生了反转。事件中问题确实存在,但当事人的身份透明导致了这一反转。\par
	情绪化也网络舆情的一个特点。情绪化体现在人们对于事情的判断产生明显偏见,甚至无厘头地支持或反对。面对网络上发生的事件,弱势群由于体无能为力,更容易得到人们的同情、声援;某些群体由于对政府行为的不满也会倾向于支持除政府或专家以外的观点,也更容易被不良分子诱导。例如,在2022年11月末发生的新疆小区火灾,一定程度上存在由于不合理的封控导致人们对政府的不满。\par
	重要性是网络舆情的又一个特点。重要性体现在网络舆情管理首先关系到网络秩序,并在很多情况下对政府公信力、网络的公序良俗产生较大影响。许多情感化、片面化的内容不能直接作为法律上的处理依据,因此网络舆情的管理解决也是困难的。\par
	%-----------------------------------------------
	
	%	第二章
	\chapter{网络舆情管理的现状}
	
	%-------------------------------------------------------------
	\section{2023年上半年舆情事件特点总结}
	一、现象级舆情社会心理烙印深刻;\par
	二、舆情发生源头更加多样;\par
	三、反转舆情严重伤害政府公信力;\par
	四、朝令夕改式舆情凸显舆评前置重要性;\par
	五、媒体异地监督流量新闻成为舆情助推器;\par
	六、关联叠加效应使类型化舆情压力倍增;\par
	七、形象损害类舆情形成追责模式;\par
	八、正面宣传翻车屡见不鲜;\par
	九、文体演艺活动领域风险显而易见;\par
	十、眼见为实直观刺激提升舆情伤害程度;\par
	十一、公众认知中的法律模糊地带易成焦点。\upcite{unofficial1}
	\section{其他现状}
	2023年下半年上述特点依然存在,维持舆情热点“常看常新”的特点。媒体流量持续发挥较大作用,产生了两方面影响。一方面冷门消息借此得以大众化;另一方面,舆论带有争议性的引导可能导致过度情绪化、片面化的舆情形势。
	%-----------------------------------------------
	
	%	第三章
	\chapter{网络舆情管理方案}
	由于网络舆情的重要性,正确的网络舆情管理方法必不可少。

	%-------------------------------------------------------------
	\section{简单对策}
	从根源上,网络舆情通过互联网传播,扼杀舆情传播是不切实际的,正如“纸里包不住火”;肆意的传播也会导致被不法分子利用的后果。更好的方法是尽可能公开细节,以避免片面的舆情导向。在极其特别的情况下,应以维护网络秩序为首要和核心的目标,尽可能追求公平正义。\par
	从人们认识的角度,正确的认知、良好的逻辑可以帮助人们更加理性的看问题,更加有利于将舆情导向正轨。因此应该在社会范围内,借助互联网开展科普,尽可能减少认识盲区招致的舆情反转。\par
	从政府、媒体等有额外“话语”的机构角度,首先应保证内容的正确性,尽可能避免出于维护网络秩序的名义传播错误消息的情况。上述情况的危害在于理想与实际脱轨导致的“愚民”到一定程度会被揭开,导致公信力的巨大危机。因此在这方面的权衡也应该谨慎考虑。
\makecover

\begin{abstract}{\LaTeX; ; 课程论文; 网络舆情}
    网络舆情是由于各种事件的刺激而产生的、通过互联网传播的、人们对于该事件的所有认知、态度、情感和行为倾向的集合。它具有片面性强、情绪化强、十分重要的特点。目前的现状是:“网络舆情是由于各种事件的刺激而产生的、通过互联网传播的、人们对于该事件的所有认知、态度、情感和行为倾向的集合”。对此应当采取透明处理、积极应对、拓宽知识面等方案。
\end{abstract}


\begin{abstractEng}{\LaTeX; Course article;Network public opinion }
    Network public opinion is a collection of people's perceptions, attitudes, emotions, and behavioral tendencies towards an event, which is stimulated by various incidents and disseminated through the internet. It is characterized by strong subjectivity, emotional intensity, and significant importance. The current situation is that "network public opinion is a collection of people's perceptions, attitudes, emotions, and behavioral tendencies towards an event, which is stimulated by various incidents and disseminated through the internet." In response to this, transparent handling, proactive measures, and broadening knowledge are recommended.
\end{abstractEng}


\tableofcontents